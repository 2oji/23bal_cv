%%%
% Tex file
%
%
%%%


%
% About
\vspace{-3pt}%
%
\MainHeading{About}\footnotesize{%
   Proficient in Boot Loader, Linux programming and OS concepts is matched by my expertise in porting Linux onto ARM based devices, as well as my ability to effeciently deliver new features, resolving complex bugs, analysing the scenarios, use-cases, handle board bring-up activities, including bootloader compilation and downloading on the target board.
   \newline%
   With a strong background in programming and data structures, I have a good understanding of Linux internals and strong debugging skills alog with expertise on I2C, SPI, UFS, MDIO, Ethernet stack and Layer 2 of OSI model.
   \newline%
   Actively engaged with relevant stakeholders, customers, CE teams to driver features and issues to end.
}
%
%
%
\vspace{-12pt}%
% Experience
%
\MainHeading{Technical Skills}
\vspace{-\topsep}\vspace{2pt}
\footnotesize {
   \begin{itemize}[label={\textendash}]\itemsep\isep
      %\item \noindent Having a 3 year's experience as embedded software developer on LINUX operating system.
      %\item \noindent Involved in customer gathering and proposed efficient solutions.
      \item \noindent Working on Linux kernel and Zephyr OS.
      \item \noindent Experience in MAC driver for ethernet at L2.
      \item \noindent Knowledge of Network Pakcet flow and with TCP/IP layer.
      \item \noindent Planned the software architecture as per customer's requirement.
      \item \noindent Done the Board Bring up, porting of u-boot, device-tree and linux.
      \item \noindent Experience in debugging of interface like DDR, NAND, I2C, SPI and UART.
      \item \noindent Experience in networking for L2, priority scheduling, Round Robin scheduling.
      \item \noindent Experience in TCP/IP, UDP, layer 2, layer 4 and complete network packet flow.
      \item \noindent Experience in Ethernet driver, socket buffers i.e. skb packets to prioritize UDP packets.
      \item \noindent Write slave drivers for I2C and SPI for micro-processor, as Linux kernel doesn’t provide support for slave mode.
      \item \noindent Write linux device driver to access DDR for firmware upgrade \& fail-safe mechanism.
      \item \noindent Experience with interfaces like NAND, EMMC, Ethernet, USB, DDR, NOR, Sd-Card, Bluetooth.
      \item \noindent Experience in debugging I2C, SPI, UART slave devices and at linux user space.
      \item \noindent Customized boot-loader and Linux operating system for embedded platforms.
      \item \noindent Experienced with i.MX 6 sabresd/sololite, Mini2440, 
      \href{http://www.radiumboards.com/TI\_TDA2x\_Based\_xCAM\_Platform.php}{\textbf{\underline{TI's TDA2x}}},
      \href{http://www.nxp.com/products/reference-designs/qoriq-ls1021a-iot-gateway-reference-design:LS1021A-IoT}{\textbf{\underline{LS1021A-IOT}}} boards.
         \item \noindent Experience with processors like TI's omap, Freescale's i.MX 6, Qualcomm's IPQ4029 \& IPQ4029.
         \item \noindent Ported the Marvell 88E6352 chip in linux and u-boot based on Distributed Switch Architecture for Ethernet.
         %\item \noindent Experience with SDK: Yocto, Openwrt, VISION\_SDK.
         %\item \noindent Having a good analysis and debugging skills.
         \end{itemize}
\vspace{3pt}
}

%
%
%
\MainHeading{Experience}
%
%
%
% New Company
%
%
\MainSubHeading{\CompanyV{}}{| Staff Engineer}
\MainSubHeadingNote{\CompanyVexp{}}
\RoleDescription{%
   Handling storage technology for all the AUTO SOC's primarily for UFS \& its dependency across subsystem like PBL, PMIC, IMEM, DDR and boot. Optimize the existing feature to be more robust and more efficient.
%   Overall Non-HLOS Storage technology POC's for all the AUTO SOC's primarily handling UFS technology \& its dependency across subsystem like PBL, PMIC, IMEM, DDR and Boot \textendash{} delivering optimizations related to boot, flashing, new features along with debugging Production line, Customer \& Blocker issues. \newline
   Effectively addressing stakeholders across management, CE, Customers and internal teams for the timelines.%
}
\vspace{-\topsep}\vspace{2pt}
\begin{itemize}\itemsep\isep
\footnotesize {
      %
      %
      \item \noindent {\textbf{Optimize UFS probe}}, Designed feature to optimize the UFS probe of Linux kernel to improve the performance by 90msec. UFS PHY is already initalized at PBL stage and same state can be leveraged to Linux kernel i.e. to skip the PHY operation. Removed registration to put the UFS device in sleep mode as part of UEFI Exit Boot services, as when HLOS is loaded UFS remains in active state. Implemented backend apis to set voltage, clock and reset for UFS driver in backend GearVM driver to support this feature.
      %
      %
   \item \noindent \textbf{UFS - Boot KPI} Optimize the boot flow for improvements \textemdash{} by 200msec \\
      \BAspaceBullet UFS partial Init               \BAspaceBullet Remove retry of UFS2.x \\
      \BAspaceBullet Remove read in PWM Gear        \BAspaceBullet UFS read increased to 256MB data   \\
      \BAspaceBullet PMIC init parallel to UFS
      \item \noindent Design Mission \& Non-mission mode feature for UEFI - a Licensed programme where subscribed AUTO OEMs will get whole feature as per JEDEC spec to validate the UFS device in UEFI. Implemented Health Customers test-cases 
%%      \item \noindent {\textbf{UFS successful bringup of LeMans}}, Company's first ever AUTO SOC i.e. SA8255.
%%      \vspace{-\topsep}\vspace{0pt}    % Remove any vertical spaced added by itemsep
%%%
%%      \begin{itemize}\itemsep\isep
%%      \item \noindent Analyze \& Quickly issue Clock fix for UFS to un-block Bring up.
%%      \end{itemize}\vspace{-1mm}%
%%%
      \item \noindent Developed the \textit{provisioning for UFS and descriptor} access apis.
      \item \noindent Supported UFS Write Booster parameters to enhance flashing.
      \item \noindent Quest test-cases in UEFI to facilitate UFS validations.
      \item \noindent Leading Design/Development for UFS across teams like PMIC, Clocks \& Boot.
      \item \noindent Supported CE teams across the globe having different time zones for storage related issues.
      \item \noindent Collaborating in Design discussions, Documentation, code reviews \& technical issues.
      \item \noindent Actively engaged in customer issues and driving from storage side.
      %
      % PA_AvailTxDataLanes and PA_AvailRxDataLanes to 2 in PBL and give it a try.
      \item \noindent {\textbf{UFS Phy Skip}} Optimize the KPI by 100 msec by leveraging the UFS PHY configurations of PBL. Worked along with PBL team for the proper UFS PHY configurations. Resolved the bug in XBL while enabling 2 lanes as PA\_Avail Lanes parameter is not set to 2 which is originated from PBL.
      %
      %
      \item \noindent {\textbf{Scatter Gather Feature}} Designed and Implemented the Scatter Gather functionality for UFS read operation where data from UFS LBA is loaded at various DDR memory addresses to support loading of boot images. This has eliminated the multiple read operation with single read operation which has resulted in 70\% improvement in KPI.
      %
      % LeMans: PHY Init fails as PHY is configured HSG4 and 2 lane settings. SERDES start is failing as issue is with Clocks - Here Skip the PHY setting to use the PBL one and un-block other teams.
      \item \noindent {\textbf{UFS successful bringup of LeMans}}, Company's first ever AUTO SOC i.e. SA8255. Meticulously planned for all the use-cases of UFS for XBL, UEFI, flashing tools and its challenges, limitations and dependencies. Resolved critical issues related to clock while enabling 2 lanes for UFS in XBL. Release workaround to unblock other teams till the proper fix.  Successfully done the UFS Bring up number of Auto SOC's i.e. SA8775, SA8620, SA8797.
      %
      %
      \item \noindent {\textbf{UFS Cache BUG per 50K iterations}}, Several UFS commands are failing like to get bootable LUN, geometry, config and unit descriptors. Each failure has different use-case. Analayzed all the failure errors where commands are successfully received at UFS device which futher processed the right response to Host controller. Here, Host observed the mismatch in command sent and received response which marks failure of all these use-cases. Issue is with cache memory where Host is reading the stale response that cause this mismatch. Flushing the cache memory resolved this issue.
      %
      %
      \item \noindent {\textbf{First UFS read cmd fails after PMIC init}}, Analyzed and solved this critical bug, where UFS read is failing after PMIC initialization and reported error in PA DL layers. Adding delay and re-issue same command would have averted this issue. Furhter read operations are passing. Here, observed that ufs read cmd is passing with PON
\framebreak
      %
      % UEFI - remove SSU2 commands and do not put device in sleep mode + clock gating.
      % HLOS saw UFS down so it has to re-init the while PHY sequence.
      % \item \noindent Re-designed the UFS driver in UEFI, GearVM and in HLOS to improve the boot time by 90msec. Here, UEFI has to skip issuing hibernate command as part of exit boot services and HLOS has to skip the PHY initialization sequence as UFS device is in active state. So, HLOS has to do partial initialization to initialize its stack which requires support from GearVM for individual apis for voltage, clock, reset and PHY configurations.
}
\end{itemize}\vspace{-\topsep}%
sequence and failed after that. Observed the likely events that can affect UFS so found that PMIC initialization is driving UFS voltage and strengths. Furhter found that PSI values are different from PON sequence and can crate jitters in first UFS read operation. Issue routed to PMIC team.
\sectionsep

%
%
% Company
%
%
\MainSubHeading{\CompanyIV{}}{| Technical Leader}
\MainSubHeadingNote{\CompanyIVexp{}}

%%\hspace{0.25cm}\begin{tabularx}{\textwidth}{>{\hsize=0.67\hsize}X >{\hsize=1.3\hsize}X} % Total width = \linewidth. Mix of types.
\vspace{-\topsep}\vspace{2pt}    % Remove any vertical spaced added by itemsep
% One option is to use the commands \setlength and \arraystretch to change the horizontal spacing (column separation) and the vertical spacing (row separation) respectively; for example:
\begingroup
\setlength{\tabcolsep}{5pt} % Default value: 6pt i.e 0.03\textwidth = 5 pt
\indent\begin{tabularx}{\textwidth}{p{0.30\textwidth} p{\RIGHTCOLUMNWIDTH}} % Total width = \linewidth. Mix of types.
%    \hline
      Determining AUTOSAR SWS specifications for embedded interfaces required for a new microcontroller, ensuring critical timing requirements and implementation, validation of RADAR frequency band test coverage.
         &
        \vspace{-\topsep}\vspace{2pt} \begin{myitemize}
    % Remove any vertical spaced added by itemsep
         \footnotesize {
            \item \noindent Defining the AUTOSAR SWS specifications for
                           \href{https://www.autosar.org/fileadmin/standards/R22-11/CP/AUTOSAR_SWS_PortDriver.pdf}{Port},
                           \href{https://www.autosar.org/fileadmin/standards/R22-11/CP/AUTOSAR_SWS_SPIHandlerDriver.pdf}{SPI},
                           \href{https://www.autosar.org/fileadmin/standards/R24-11/CP/AUTOSAR_CP_SWS_I2CDriver.pdf}{I2C},
                           interfaces for new microcontroller.
            \item \noindent Analyzed the requirement to initiated first CAN message in 50msec after POR.
            \item \noindent Integrated the Radar chipset over I2C bus with S32R series microcontroller.
            \item \noindent Implemented the test-cases covering full RADAR frequency band from 76 GHz to 81 GHz with variations in sweeping window.
            \item \noindent Analyzed the generated matlab plot for the RADAR tests.
            }
         \end{myitemize}\vspace{-\topsep}%
\vspace{-5pt}%
\end{tabularx}
%% \indent\begin{tabularx}{\textwidth}{p{0.30\textwidth} p{\RIGHTCOLUMNWIDTH}} % Total width = \linewidth. Mix of types.
%% %    \hline
%%    Worked Responsible for device driver evelopment on Ethernet, including porting and tailoring per customer needs. & \vspace{-\topsep}\vspace{2pt} \begin{myitemize}
%%     % Remove any vertical spaced added by itemsep
%%          \footnotesize {
%%     \item \noindent Integrate phy-less port \href{https://www.marvell.com/switching/link-street/}{\textbf{Marvell 88E6352 Switch}} with \href{http://www.nxp.com/products/reference-designs/qoriq-ls1021a-iot-gateway-reference-design:LS1021A-IoT}{\textbf{\underline{LS1021A-IOT}}} over PHY-less port.
%%             \item \noindent High Priority scheduling for selected TCP/UDP packets with Round Robin mechanism.
%%             \item \noindent Understand the complete network packet flow, layer 2 and working of Network Switch.
%%             \item \noindent Experience in Ethernet driver, socket buffers i.e. skb packets to prioritize UDP packets.
%%             \item \noindent Tailored device tree and gianfar linux driver.
%%             }
%%          \end{myitemize}\vspace{-\topsep}%
%% \end{tabularx}
\endgroup
\sectionsep       %%% Followed per Deedy resume
%
%
%
% Company
%
%
\vspace{-9pt}
\MainSubHeading{\CompanyIII{}}{| Technical Leader}
\MainSubHeadingNote{\CompanyIIIexp{}}
%%  \RoleDescription{%
%%     Gained experience on Ethernet, Wireless domain, linux kernel and system applications.
%%  }
%%  \vspace{-\topsep}\vspace{2pt}    % Remove any vertical spaced added by itemsep
%%  \begin{itemize}\itemsep\isep
%%  \footnotesize {
%%     \item \noindent OTA upgrade - Implemented failsafe mechanism.
%%     \item \noindent Integrated Atheros PHY with Marvell SOC \textendash{} MDIO protocol.
%%     \item \noindent Wireless Client aborts communication in 5GHz B.W. upon RADAR detection on \\ current DFS channel and raise CAC to station - first ever design for Client.
%%  }
%%  \end{itemize}\vspace{-\topsep}%
\begingroup
\setlength{\tabcolsep}{5pt} % Default value: 6pt i.e 0.03\textwidth = 5 pt
\indent\begin{tabularx}{\textwidth}{p{0.30\textwidth} p{\RIGHTCOLUMNWIDTH}} % Total width = \linewidth. Mix of types.
%    \hline
   Handles OTA firmware update and system level programming, Integrated Atheros PHY and provided software support to comply with  passing criteria.
   Gained experience on Ethernet, Wireless domain, linux kernel and system applications.
         &
        \vspace{-\topsep}\vspace{2pt} \begin{myitemize}
    % Remove any vertical spaced added by itemsep
    \footnotesize {
    \item \noindent OTA upgrade - Implemented fail-safe recovery mechanism to update the boot binaries if OTA fails for boot binaries.
    \item \noindent Ported and Integrated Atheros PHY with Marvell SOC \textendash{} MDIO protocol in linux kernel.
    \item \noindent Implemented use-case on the Wireless Client to aborts communication in 5GHz B.W. upon RADAR detection on current DFS channel and raise CAC to station - first ever design for Client. This use-case is required for Safety approvals at the Airport RADAR is detected frequently in 5GHz frequency range.
    }
   \end{myitemize}\vspace{-\topsep}%
\end{tabularx}
\sectionsep       %%% Followed per Deedy resume
%
%
% Company
%
%
%% \MainSubHeading{\CompanyII{}}{| Senior Engineer}
%% \MainSubHeadingNote{\CompanyIIexp{}}
\vspace{-9pt}
{\noindent \large{\textsc{\textbf{\CompanyII} |Senior Engineer}}}\newline%
{\noindent \color{ltgrey}\small{\textit{\CompanyIIexp{}}}}\newline%
%% \RoleDescription{%
%%    Worked in boot domain and experienced with Jenkins.
%% }
%% \vspace{-\topsep}\vspace{2pt}    % Remove any vertical spaced added by itemsep
%% \begin{itemize}\itemsep\isep
%% \footnotesize {
%%    \item \noindent Automate flashing \& loading of Wi-Fi FW binary from HOST over TFTP in boot terminal.
%%    \item \noindent Jenkins to automate Ethernet test-cases
%% }
%% \end{itemize}\vspace{-\topsep}%
\begingroup
\setlength{\tabcolsep}{5pt} % Default value: 6pt i.e 0.03\textwidth = 5 pt
\indent\begin{tabularx}{\textwidth}{p{0.30\textwidth} p{\RIGHTCOLUMNWIDTH}} % Total width = \linewidth. Mix of types.
%    \hline
   Automate the loading of binary at boot terminal and build understanding on Jenkins across targets for nightly tests.
         &
        \vspace{-\topsep}\vspace{2pt} \begin{myitemize}
    % Remove any vertical spaced added by itemsep
    \footnotesize {
      \item \noindent Implemented command in boot terminal, that will first initialize localhost IP configurations and loads the Wi-Fi FW binary from locahost over TFTP and start flashing at the required address.
      \item \noindent Understand the Jenkins framework and scale the existing framework to new targets.
    }
   \end{myitemize}\vspace{-\topsep}%
\end{tabularx}
\endgroup
\sectionsep
%
%
\vspace{-5pt}%
%
%
% Company
%
%
\MainSubHeading{\CompanyI{}}{| Senior Engineer}
{\noindent \color{ltgrey}\small{\textit{\CompanyIexp{}}}}\newline%
%\MainSubHeadingNote{\CompanyIexp{}}
 Responsible for the complete embedded system, from understanding customer requirements, developing feature, resolving bugs/issues and end-to-end validations.
 Resolved numerous complex issues while porting and in the bring-up phase in u-boot, linux kernel and system level applications.
\newline %%
\begin{minipage}[b]{\textwidth}%%% BUG:todo as few items are coming to next line
  \setlength{\textwidth}{20cm}
    \footnotesize {%
       \begin{itemize}[itemindent=0pt,rightmargin=0pt, label={\textendash}, nosep]
          \item Board Bring UP of multiple targets like TI's DM388, Qualcomm, TDA2xx and Freescale i.MX6 series.
          \item Porting and optimizations in u-boot, linux kernel, device-tree \& System level applications.
          \item Implemented IPC's, Multi thread and Socket Programming.
          \item Written applications for hostapd, udhcpc, wpa\_supplicant, nodejs, boa server to meet use-cases.
          \item Experienc in debugging interfaces like I2C, SPI, Ethernet, MDIO, Router, Bluetooth, DDR and integrating memories i.e. NAND, eMMC with Host controller.
          \item Experience in MAC driver for ethernet at L2.
          \item Knowledge of Network Packet flow and with TCP/IP layer.
          \item Experience interface like DDR, NAND, I2C, SPI and UART.
          \item Experience in networking for L2, priority scheduling, Round Robin scheduling.
          \item Experience in TCP/IP, UDP, layer 2, layer 4 and complete network packet flow.
          \item Experience in Ethernet driver, socket buffers i.e. skb packets to prioritize UDP packets.
          \item Write slave drivers for I2C and SPI for micro-processor, as Linux kernel doesn’t provide support for slave mode.
          \item Write linux device driver to access DDR for firmware upgrade \& fail-safe mechanism.
          \item Experience with interfaces like NAND, EMMC, USB, DDR, NOR, Sd-Card, Bluetooth.
          \item Experience in debugging I2C, SPI, UART slave devices and at linux user space.
          \item Ported the Marvell 88E6352 chip in linux and u-boot based on Distributed Switch Architecture for Ethernet.
   \end{itemize}\vspace{6pt}%
    }%
\end{minipage}
%
%
%
%%  \vspace{4pt}
%%  \footnotesize {
%%  \hspace{0.25cm}{\textbf{Marvell Gigabit Ethernet Switch Integration}}\newline
%%     Integrate the i.MX6 sololite \& TI's DM8168 processor with \href{https://www.marvell.com/switching/link-street/}{\textbf{Marvell 88E6352 switch}} having 7 ports, 1 SERDES and 5 PHY's with gigabit ethernet support. Marvell switch intergarted with processor directly with PHY less PORT. i.MX6 sololite configured in RMII configuration and DM8168 configured GMII mode. \newline
%%     \emph{Challenge} is to understand the Marvell switch integration with controller.
%%  %, learned about MDIO framework
%%        \newline
%%        \emph{Responsibilities:}
%%        \begin{itemize}\itemsep \isep
%%           \item Understands controller engagement with switch over phy-less port i.e PORT 6 of switch.
%%           \item At u-boot, PHY support is added and configure the processor in RMII mode for i.MX6.
%%           \item Resolved the ping bug, by adding delay at phy side.
%%           \item In linux, ethernet driver is modified to keep the link permanently up.
%%        \end{itemize}
%%  }
\begingroup
\setlength{\tabcolsep}{5pt} % Default value: 6pt i.e 0.03\textwidth = 5 pt
\indent\begin{tabularx}{\textwidth}{p{0.30\textwidth} p{\RIGHTCOLUMNWIDTH}} % Total width = \linewidth. Mix of types.
%    \hline
    {\textbf{Marvell Gigabit Ethernet Switch Integration}} & \emph{Responsibilities} \\%%
    Integrate the i.MX6 sololite \& TI's DM8168 processor with \href{https://www.marvell.com/switching/link-street/}{\textbf{Marvell 88E6352 switch}} having 7 ports, 1 SERDES and 5 PHY's with gigabit ethernet support. Marvell switch intergarted with processor directly with PHY less PORT. i.MX6 sololite configured in RMII configuration and DM8168 configured GMII mode.  &
        \vspace{-\topsep}\vspace{2pt} \begin{myitemize}
    % Remove any vertical spaced added by itemsep
    \footnotesize {
         \item Understands MDIO protocol to engage with Marvell Switch as need to integrate PHY less port of switch.
         \item Configure the Marvell switch to use phy less port, read its device id.
         \item Implemented Switch driver at u-boot and added necessary PHY support to configure the processor in RMII mode for i.MX6.
         \item Resolved the ping issue by adding delay for the rx line on the host side of PHY configurations.
         \item Modified the linux ethernet driver to disable AUTO PHY negotiations to keep link always up for 1000 Gbps
    }
   \end{myitemize}\vspace{-\topsep}%
\end{tabularx}
\endgroup
%
%
\vspace{-4pt}
%
%
%% \hspace{0.25cm}\href{https://www.qualcomm.com/products/ipq4028}{\textbf{Wireless AP-CPE }}%\hfill Duration: 2 months
%%    %\item {\textbf{Wireless AP-CPE}}%\hfill Duration: 5 months
%% %    \emph{Platform : Linux\hfill Build-SDK: openwrt\\ Role : Software Developer} \\[-0.6cm]
%%    % \makebox[\linewidth][s]{\emph{CPE establish Ethernet 1Gig connection with i.MX 6 based board and sends it's data to AP board over wireless 5Gig channel.}}
%%    Based on Dakota reference board having Qualcomm IPQ4028 chipset and integrated with BLE module, SPI-Nor flash, USB-OTG, Ethernet Atheros Phy, 2x2 Wi-Fi 5G and 2G radios. Provides various applications as Access points, Home Routers and as Customer Premises Equipment i.e. CPE.
%%    \newline
%% 	\emph{Responsibilities:}
%% 	\begin{itemize}\itemsep \isep
%% 	\item Board Bring-Up.
%% 	\item Porting of Linux, Device-Tree , U-Boot and Board Bring-Up activities.
%% 	\item Removed the support of 5 Port Phy i.e. QCA8075.
%% 	\item Provide support for Atheros 8033 phy in linux in u-boot and kernel.
%% 	\item Configured the uart interface for BLE i.e. CSR1012 module.
%% 	\item Bug-Fix for auto-negotiation of PHY i.e. provide support of 10/100MBps.
%% 	\item System level implementation to Dynamically Configured as Access Point or as CPE. % based on gpio.
%% 	\end{itemize}
\begingroup
\setlength{\tabcolsep}{5pt} % Default value: 6pt i.e 0.03\textwidth = 5 pt
\indent\begin{tabularx}{\textwidth}{p{0.30\textwidth} p{\RIGHTCOLUMNWIDTH}} % Total width = \linewidth. Mix of types.
%    \hline
    \href{https://www.qualcomm.com/products/ipq4028}{\textbf{Wireless AP-CPE }} & \emph{Responsibilities} \\%%
    Based on Dakota reference board having Qualcomm IPQ4028 chipset and integrated with BLE module, SPI-Nor flash, USB-OTG, Ethernet Atheros Phy, 2x2 Wi-Fi 5G and 2G radios. Provides various applications as Access points, Home Routers and as Customer Premises Equipment i.e. CPE.
         &
        \vspace{-\topsep}\vspace{2pt} \begin{myitemize}
    % Remove any vertical spaced added by itemsep
    \footnotesize {
       \item Board Bring-Up.
	    \item Porting of Linux, Device-Tree , U-Boot and Board Bring-Up activities.
       \item Removed the support of 5 Port Phy i.e. QCA8075.
	    \item Provide support for Atheros 8033 phy in linux in u-boot and kernel.
	    \item Configured the uart interface for BLE i.e. CSR1012 module.
	    \item Bug-Fix for auto-negotiation of PHY i.e. provide support of 10/100MBps.
	    \item System level implementation to Dynamically Configured as Access Point or as CPE. % based on gpio.
    }
   \end{myitemize}\vspace{-\topsep}%
\end{tabularx}
\endgroup
%
%
\vspace{-2pt}
%
%%
%%
%% \hspace{0.25cm}{\textbf{Packet Prioritization}}\newline
%% 	Based on the NXP's
%%         \href{http://www.nxp.com/products/reference-designs/qoriq-ls1021a-iot-gateway-reference-design:LS1021A-IoT}{\textbf{\underline{LS1021A-IOT}}}
%% board. UDP Network packets sent on high priority through ethernet. Locate the UDP packets having specific destination port and mapped them to high priority queues with the support of existing eTSEC hardware.
%%         \newline
%%         \emph{Responsibilities:}
%%         \begin{itemize}\itemsep \isep
%%             \item Implemetation done in device tree and gianfar driver.
%%             \item Capture the socket buffers for UDP packets, having a specific destination port at layer 4.
%%             \item Mapped the socket buffers to high priority queues.
%%             \item Enable the High Priority Scheduling, supported by eTSEC hardware at layer 2.
%%             \item Enables the Weight Round Robin mechanism to send mapped packet at Egress port.
%%         \end{itemize}
%%
%%
\begingroup
\setlength{\tabcolsep}{5pt} % Default value: 6pt i.e 0.03\textwidth = 5 pt
\indent\begin{tabularx}{\textwidth}{p{0.30\textwidth} p{\RIGHTCOLUMNWIDTH}} % Total width = \linewidth. Mix of types.
%    \hline
    \textbf{Packet Prioritization} & \emph{Responsibilities} \\%%
        \href{http://www.nxp.com/products/reference-designs/qoriq-ls1021a-iot-gateway-reference-design:LS1021A-IoT}{\textbf{\underline{LS1021A-IOT}}}
          board. UDP Network packets sent on high priority through ethernet. Locate the UDP packets having specific destination port and mapped them to high priority queues with the support of existing eTSEC hardware.
         &
        \vspace{-\topsep}\vspace{2pt} \begin{myitemize}
    % Remove any vertical spaced added by itemsep
    \footnotesize {
             \item Implemetation done in device tree and gianfar driver.
             \item Capture the socket buffers for UDP packets, having a specific destination port at layer 4.
             \item Mapped the socket buffers to high priority queues.
             \item Enable the High Priority Scheduling, supported by eTSEC hardware at layer 2.
             \item Enables the Weight Round Robin mechanism to send mapped packet at Egress port.
    }
   \end{myitemize}\vspace{-\topsep}%
\end{tabularx}
\endgroup
%
%
\vspace{-2pt}
%
%% \hspace{0.25cm}{\href{https://youtu.be/xr9ez-oayiI}{\textbf{ADAS Multi-sensor fusion Camera}}}\newline
%%  	Based on TI's TDA2xx OMAP processor to captures data from various sensors and fuses all that data to produce a unique frame for Automotive industry for ADAS applications. It is also integrated with DaVinci Digita Media processor i.e DM388, for high resolution images through board to board connector. TDA2xx Processor fuses the data from sensor and DM388 processor to produce a video which is streamed on HDMI.
%%         \newline
%%         \emph{Responsibilities:}
%%         \begin{itemize}\itemsep \isep
%%         \item U-boot and Linux Kernel firmware porting for DM388 and Tda2xx
%%         \item Configured SPI bus for Mems Thermal Sensor in Tda2xx.
%%         \item Write the I2C slave driver for DM388 as linux kernel don’t provide support for I2C slave mode.
%%         \item Modify the SPI driver for DM388 to support slave configuration.
%%         \item DDR Performance with Cachebench utility.
%%         \item Testing of Ethernet interface i.e.\ validation of 1 GigaBit link at linux user space with iperf and ethtool.
%%         \item Testing of USB 3.0 interface at linux user space.
%%         \item Nand flash: Partitionin of NAND flash.
%%         \end{itemize}
%% \vspace{-\topsep}\vspace{2pt}    % Remove any vertical spaced added by itemsep
% One option is to use the commands \setlength and \arraystretch to change the horizontal spacing (column separation) and the vertical spacing (row separation) respectively; for example:
\begingroup
\setlength{\tabcolsep}{5pt} % Default value: 6pt i.e 0.03\textwidth = 5 pt
\indent\begin{tabularx}{\textwidth}{p{0.30\textwidth} p{\RIGHTCOLUMNWIDTH}} % Total width = \linewidth. Mix of types.
%    \hline
    \textbf{ADAS Multi-Sensor Fusion Camera} & \emph{Responsibilities} \\%%
	Based on TI's TDA2xx OMAP processor to captures data from various sensors and fuses all that data to produce a unique frame for Automotive industry for ADAS applications. It is also integrated with DaVinci Digita Media processor i.e DM388, for high resolution images through board to board connector. TDA2xx Processor fuses the data from sensor and DM388 processor to produce a video which is streamed on HDMI.
         &
        \vspace{-\topsep}\vspace{2pt} \begin{myitemize}
    % Remove any vertical spaced added by itemsep
    \footnotesize {
          \item U-boot and Linux Kernel firmware porting for DM388 and Tda2xx
          \item Configured SPI bus for Mems Thermal Sensor in Tda2xx.
          \item Write the I2C slave driver for DM388 as linux kernel don’t provide support for I2C slave mode.
          \item Modify the SPI driver for DM388 to support slave configuration.
          \item DDR Performance with Cachebench utility.
          \item Testing of Ethernet interface i.e.\ validation of 1 GigaBit link at linux user space with iperf and ethtool.
          \item Testing of USB 3.0 interface at linux user space.
          \item Nand flash: Partitionin of NAND flash.
    }
   \end{myitemize}\vspace{-\topsep}%
\end{tabularx}
\endgroup
