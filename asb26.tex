%%%%%%%%%%%%%%%%%%%%%%%%%%%%%%%%%%%%%%%%%%%%%%%%%%
%
% Latex Template - Tex file
% Compile with xelatex
% Version 1.0
% Shared under GPLv3
%
%%%%%%%%%%%%%%%%%%%%
%
%
%%%%%%%%%%%%%%%%%%%%
%
%
%%%%%%%%%%%%%%%%%%%%%%%%%%%%%%%%%%%%%%%%%%%%%%%%%%
%

\documentclass[a4paper]{article} % use "letterpaper" if you're in the US

%%%%%
% Load Geometry package after \LoadClass{article} to make effect
%
%\usepackage[top=0.75in, bottom=0.75in, left=0.55in, right=0.85in]{geometry}
\usepackage[hmargin=1.25cm, vmargin=0.75cm]{geometry}
%%%%%

%%%
% Dump data for checking - \blindtext[1]
% \blindtext = paragraph of dummy text
% \Blindtext = multiple paragraphs (one page long) of dummy text
% \Blinddocument = entire document, If you specify documentclass
%      as report then \Blinddocument will generate the document
%      with multiple chapters, sections, subsections, paragraphs,
%       and itemized lists.
\usepackage{blindtext}
\usepackage{lipsum}              % Dummy data - lipsum[2-4]
%%%

%%% Load graphics pacakge
\usepackage{graphicx}            %% For linkedin icons, github
\usepackage{calc}                %% To calculate the localLocPin
\usepackage{tabularx}
%%%
% all later pages the natural height of the material on that page
% no rubber vertical lengths will be stretched
%
\usepackage{ragged2e}
%\textheight=9.75in
\raggedbottom

%%% Symbols
% fontawesome, marvosym, ifsym few pkgs
%
%
\usepackage{fontawesome}         %% For linkedin icons, github

%%%
%
%
\usepackage{xcolor} % used for colouring and underline the links.
\usepackage{hyperref} % used to generate links i.e mailto:xxxx@ymail.com
\hypersetup{
   colorlinks=true,
   linkcolor=blue,
   filecolor=magenta,
%  urlcolor=cyan,
   urlcolor=blue,
   linkbordercolor=red
}
\definecolor{ltgrey}{gray}{0.50}
\definecolor{black}{RGB}{0, 0, 0}

%%%
%%%
% Load Confidential data
%%%%%%%%%%%%%%%%%%%%%%%%%%%%%%%%%%%%%%%%%%%%%%%%%%
%
% Font select - Tex file
% Compile with xelatex
% Version 1.0
% Shared under GPLv3
%
%%%%%%%%%%%%%%%%%%%%
%
%
%%%%%%%%%%%%%%%%%%%%
%
%
%%%%%%%%%%%%%%%%%%%%%%%%%%%%%%%%%%%%%%%%%%%%%%%%%%
%

%%% Fonts & Encodings
%
%

\def\ChooseFont{UniversNextPro}
\def\FontURWPalladio{URWPalladio}
\def\FontTGPagella{TGPagella}
\def\FontUniversNextPro{UniversNextPro}

% URW Palladio font - https://www.tug.org/FontCatalogue/urwpalladio/
\ifx\ChooseFont\FontURWPalladio
   \usepackage[sc]{mathpazo}
   \linespread{1.05}                % Palladio needs more leading (space between lines)

   % Font encoding
   \usepackage[T1]{fontenc}         % fontenc shall be at last after selecting font
                                    % Support of 8 bit encoding 256 glyphs
                                    % Display correctly |, <, >
   \usepackage[utf8]{inputenc}      % translates various standard and other input encodings into a ‘LaTeX internal language’
                                    % Characters associated with many langugages will display correctly - MUST pkg
\fi

% Tex Gyre Pagella advanced version of URW Palladio with Math support
% This has OTF and TTF files also
\ifx\ChooseFont\FontTGPagella
   \usepackage{tgpagella}

   % Font encoding
   \usepackage[T1]{fontenc}         % fontenc shall be at last after selecting font
                                    % Support of 8 bit encoding 256 glyphs
                                    % Display correctly |, <, >
   \usepackage[utf8]{inputenc}      % translates various standard and other input encodings into a ‘LaTeX internal language’
                                    % Characters associated with many langugages will display correctly - MUST pkg
\fi

%%%
% OpenType and True Type font
%
\ifx\ChooseFont\FontUniversNextPro
   \usepackage{fontspec}

   \setmainfont{UniversNextProLightCond}[
      Path = fonts/univers_pro/ ,
      Extension = .ttf ,
      BoldFont = UniversNextProBoldCond,
      BoldItalicFont = UniversNextProBoldCondIt,
      ItalicFont = UniversNextProLightCondIt,
   ]
\fi


\IfFileExists{./data/UserPvtInfo.tex}{\input{./data/UserPvtInfo.tex}}{%
%
\newcommand{\UserName}{\textsc {\Huge 2oji}}
\newcommand{\UserExperience}{13}
\newcommand{\UserMobile}{\textsc{+1234-123-123}}
\newcommand{\UserEmailID}{\textit{2oji@github.com}}
%
%
\newcommand{\UserLocation}{\textsc{Panjab}}
\newcommand{\CompanyA}{\textsc{\large Hombre}}
\newcommand{\CurrentCompany}{\CompanyA}

}

\newcommand*{\localLocPin}{%
  \includegraphics[height=\heightof{M}]{pin}%
}

\newcommand{\MainHeading}[1]{%
    {\noindent \color{ltgrey}\normalsize{\textsc {#1}}}\vspace{6pt}\\ }

\newcommand{\MainSubHeading}[2]{%
   {\noindent \large{\textsc{\textbf{#1} #2}}}\\ }


\newcommand{\MainSubHeadingNote}[1]{%
   {\noindent \color{ltgrey}\small{\textit{#1}}}\\ }


\newcommand{\RoleDescription}[1]{%           % If need, add \vspace{-4mm} at end to adjust space
   {{\color{black}\scriptsize{#1}}} \\ }

\setlength{\tabcolsep}{0in}
\newcommand{\isep}{-2 pt}
\newcommand{\lsep}{-0.5cm}
\newcommand{\psep}{-0.6cm}
\newcommand{\sectionsep}{\vspace{8pt}}

% Renew command
\renewcommand{\labelitemii}{$\circ$}


% --- Show frame of the layout --- DEBUG
% Show layout paramters - \layout or \layout* after \begin{document}
\usepackage{layout}
% \usepackage{layouts}             % Illustrates minor details - Useful package

%%% Debug
% Show outerframe, Uncomment this   - Debug
\usepackage{showframe}

%%%%
\usepackage[draft]{flowfram}
%
% draft option will draw the bounding boxes for each defined frame.
%  Bottom right of each bounding box (except for the bounding box
%  denoting the typeblock), a marker will be shown in the form: [T:IDN;IDL]
%  -  draft option set all conditionals to true
%      \showtypeblocktrue  % Display the bounding box for the typeblock.
%      \showtypeblockfalse % Do not display the bounding box for the typeblock.
%      \showmarginstrue    % Display the bounding box for the margins.
%      \showmarginsfalse   % Do not display the bounding box for the margins.
%      \showframebboxtrue  % Display the bounding box for the frames.
%      \showframebboxfalse % Do not display the bounding box for the frames
%
% The text of the document environment will flow from one frame to the next in
%  order of definition.
%
% \newflowframe[page-list]{width}{height}{x}{y}[label]
%  - width, height - width, height of frame
%  - x, y - Position of the bottom left hand corner of the frame relative to the
%    bottom left hand corner of the typeblock.
%
%  - page-list - optional argument, indicates the list of pages for which this
%    frame is defined. A page list can either be specified by the
%    keywords: all, odd, even or none, or by a comma-separated list of either
%    individual page numbers or page ranges. If page-list is omitted, all assumed.
%
\def\LEFTCOLUMNWIDTH  {0.65\textwidth}            % 2/3 of text width
\def\RIGHTCOLUMNWIDTH {0.32\textwidth}            % 1/3 of text width
\def\SIDEBYSIDEGAP {0.68\textwidth}

\newcommand{\LeftColWdth}{0.66\textwidth}
\newcommand{\RightColWdth}{0.33\textwidth}

% define lengths to help compute positions
\newlength\titleH
\newlength\titleW
\newlength\titleY

% Create a static frame to put title in
\setlength{\titleH}{3cm}
\setlength{\titleW}{\textwidth}                 % Set textwidth
\setlength{\titleY}{\textheight}                % Set textheight
\addtolength{\titleY}{-\titleH}                 % Decrease Title Height by titleH
%  Page 1 - Title Column - Static F1
\newstaticframe[1]{\titleW}{\titleH}{0pt}{\titleY}[title]
%
%  Page 1 - Right Column - Static F2
\newstaticframe[1]{\RIGHTCOLUMNWIDTH}{\titleY}{\SIDEBYSIDEGAP}{0pt}[right]
%
%  Page 1 - Left column - Flow F 1
\newflowframe[1]{\LEFTCOLUMNWIDTH}{\titleY}{0pt}{0pt}[left]
%  Page 2 & onwards - Full page - Flow F 2
%    Define the page layout for SUBSEQUENT pages (from page 2 onwards)
%    This frame spans the full text width so that data will flow from left
%    column to this one.
\newflowframe[2]{\textwidth}{\textheight}{0pt}{0pt}[full]

%%%
% Modify flowframe attribute
% \setflowframe{ID-list}{key=value list}
%   ID-list can be found with draft option
%   Frame ID allocated in order it is defined
\setstaticframe{1}{backcolor=magenta}
\setstaticframe{2}{backcolor=green}
\setflowframe{1}{label=hi, backcolor=red}
\setflowframe{2}{backcolor=yellow}

% Space before and after bullet
\newcommand{\BAspaceBullet}{\space{\textbullet}\space}

% --- Begin Document ---
\begin{document}
%\flowframeshowlayout
%\layout                    % Debug
%

\begin{staticcontents}{1}
\begin{minipage}[b]{0.66\textwidth}        % Minipage align to bottom
%\raggedright{
   \centering{

         \UserName{} \\
         Software Architect \BAspaceBullet Embedded developer \\
         \BAspaceBullet Boot \BAspaceBullet Linux device driver \BAspaceBullet UFS  \\
         Experience \UserExperience{} Years    \\
         \faPhone{} \UserMobile{} \BAspaceBullet \space \faEnvelope{} \space \UserEmailID{}
   } %% \centering
\end{minipage}%
\hfill
%
%   Begin Left side Minipage
{
   %\noindent\fbox{%
   \begin{minipage}[b]{0.33\textwidth}        % Minipage align to bottom
         \raggedleft{

            \CurrentCompany{} \\
               Staff Engineer  \\
               {\Large{\localLocPin}}{} \UserLocation{}
         }
      \end{minipage}%
   }

\end{staticcontents}
\begin{staticcontents*}{right}
%%%
%
% Tex File
%
%%%
%
%
%
\MainHeading{Promotions \& Awards}
%%%
%  Lind 145 - 148 interferring - search for tabular - BUG1 Tabular
%%%
\small{
%\hspace*{7mm}  %% Shift the table by 1 cm
%\begin{tabularx}{0.8\textwidth}{X l}
   \begin{tabularx}{\textwidth}{X l}
      \textit{2023 Dec} & \emph{Promoted} to \textsc{Staff} Engineer.\\
      \textit{2023 Jul} & Impact Award \\
      & {~~}{\footnotesize honored by \textit{VP Enginnering}} \\        %%%{~} to add space '\,' will also add space
      \textit{2023 Jul} & Innovator of the Quarter \\
      \textit{2023 May} & Innovator of the Month \\
      \textit{2022 Apr} & Orion Award \\
      & {~~}{\footnotesize honored by \textit{Sr.Dir Engineering.}} \\
   \end{tabularx}
} %% \small

%
%
%
%
\MainHeading{Appreciations}
\footnotesize{
   \textbf{Appreciatons received across Senior Directors of Engineering for AUTO contributions} \\
   \textbullet UFS - Resolving Cache issue - \emph{very rare} \\
   \textbullet UFS \textendash{} Bring UP for LeMans. \\
   \textbullet 2nd UFS Support in UEFI \\
   \textbullet Unblocking Customers \& timely deliveries.
}

%
%
%
%
\MainHeading{Skills}
\MainSubHeading{Technical}{ }          %% BUG2 - Why new line is required here
   \textbullet Device Driver \textbullet Booting \textbullet Linux internals \\
   \textbullet I2C \textbullet SPI \textbullet MDIO \textbullet Ethernet \textbullet UFS \textbullet eMMC \\ 
   \textbullet Boot sequence - PBL to HLOS \textbullet Wireless apps\\
   \textbullet Schematic reading \textbullet System level development \\
   \textbullet Switch - MAC layer 2 \& Layer 4 \textbullet ASPICE \\
   \textbullet Web Server Programming - Boa Server \\
   \textbullet Board Bring UP activities \\
   %\textbullet \textbf{Understand Complete boot flow from PBL to HLOS loading}.
\sectionsep

%
%
%
%
\MainSubHeading{Programming}{ }          %% BUG2 - Why new line is required here
Proficient {\textit in} {\textbf C} \newline
Basic {\textit with } Bash \textbullet Python \textbullet NodeJS, Postman {\textit \scriptsize{- 4 months}} 

\sectionsep

%
%
%
%
\MainHeading{Processors | \small{\textit{ARM-A7 Architecture}}}          %% BUG2 - Why new line is required here
\textbullet TI's ADAS Processor
\textbullet OMAP series
\textbullet DM388
\textbullet SDA2XX
\textbullet Freescale i.MX6 series \\
\textbullet Qualcomm's Automobile SA8295
\textbullet SA8775 \\
\textbullet SA8255
\textbullet SA8650
\sectionsep

%
%
%
%
\MainHeading{Tools}
\textbullet T32
\textbullet JTAG
\textbullet gcc
\textbullet git
\textbullet Perforce
\textbullet SVN \\
\textbullet Vim {\textit{editor}}
\textbullet Jenkins
\sectionsep

%
%
%
%
\MainHeading{Operating System}
\textbullet Adept with Ubuntu
\textbullet Windows
\sectionsep

%
%
%
%
\MainHeading{Education}
\MainSubHeading{BTech}{\small{in} Electronics \small{\&} Communication Engineering}
\MainSubHeadingNote{Panjaab}       %% BUG3 Handle this case efficiently - Create a new command
\MainSubHeadingNote{Apr 2013, Panjaab}
\begin{itemize}\itemsep\isep
\small {
   \item \noindent Good in Mathematics
   \item \noindent Proficient in Enginnerring Drawing
}
\end{itemize}\vspace{-\topsep}%
\vspace{2mm}
%
%
%
%
\MainSubHeading{Vocational Course}{\small{in} EFY}
\MainSubHeadingNote{Jun 2011}
\sectionsep

%
%
%
%
\MainHeading{Interest}
{\small {
   \emph{Reading } Books \textbullet Articles \textbullet Poems \\
   \emph{Playing} Cards \textbullet Chess \\
   VolleyBall \textbullet Swimming
}}
\sectionsep

%
%
%
%
\MainHeading{Languages}
{\small {
   \emph{Native} Punjabi \\
   \emph{Proficient in} English \textbullet Hindi \\
   \emph{Novice in} Farsi \textbullet French
}}
\sectionsep


\end{staticcontents*}
%%%
% Tex file
%
%
%%%


%
% About
%
\MainHeading{About}

%
% Experience
%
\MainHeading{Experience}

%
%
% New Company
%
%
\MainSubHeading{\CompanyV{}}{| Staff Engineer}
\MainSubHeadingNote{\CompanyVexp{}}
\RoleDescription{%
   Overall Non-HLOS Storage technology POC's for all the AUTO SOC's primarily handling UFS technology \& its dependency across subsystem like PBL, PMIC, IMEM, DDR and Boot \textendash{} delivering optimizations related to boot, flashing, new features along with debugging Production line, Customer \& Blocker issues. \newline
   Effectively handling stakeholders across management, CE, Customers and internal teams.%
}
\vspace{-\topsep}\vspace{2pt}
\begin{itemize}\itemsep\isep
\footnotesize {
   \item \noindent \textbf{UFS - Boot KPI} improvements \textemdash{} by 200msec \\
      \textbullet UFS partial Init               \textbullet XBL - Skip UFS PHY settings    \textbullet Remove retry of UFS2.x \\
      \textbullet Remove read in PWM Gear        \textbullet UFS read increased to 256MB data   \\
      \textbullet UFS \textbf{Scatter-Gather}    \textbullet PMIC init parallel to UFS
      \item \noindent Design Mission \& Non-mission mode for UFS - a Licensed programme.
      \item \noindent {\textbf{UFS successful bringup of LeMans}}, Company's first ever AUTO SOC i.e. SA8255.
      \vspace{-\topsep}\vspace{0pt}    % Remove any vertical spaced added by itemsep
%
      \begin{itemize}\itemsep\isep
      \item \noindent Analyze \& Quickly issue Clock fix for UFS to un-block Bring up.
      %\item \noindent UFS successful bringup of LeMans, Monaco, AminiAQ \& Pinnacles.
      %\item \noindent Well planned tasks, checks, review designed for BU activities.
      %\item \noindent Prepared Back up plans for UFS prior to BU.
      \end{itemize}\vspace{-1mm}%
%
      \item \noindent Developed the \textit{provisioning for UFS and descriptor} access apis.
      \item \noindent Supported UFS Write Booster parameters to enhance flashing.
      \item \noindent Quest test-cases in UEFI to facilitate UFS validations.
      \item \noindent Leading Design/Development for UFS across teams like PMIC, Clocks \& Boot.
      \item \noindent Supported CE teams across the globe for storage related issues.
      \item \noindent Collaborating in Design discussions, Documentation, code reviews \& technical issues.
}
\end{itemize}\vspace{-\topsep}%
\sectionsep

%
%
% Company
%
%
\MainSubHeading{\CompanyIV{}}{| Technical Leader}
\MainSubHeadingNote{\CompanyIVexp{}}
\RoleDescription{%
   Responsible for device driver evelopment on Ethernet, including porting and tailoring per customer needs. %
}
\vspace{-\topsep}\vspace{2pt}    % Remove any vertical spaced added by itemsep
\begin{itemize}\itemsep\isep
\footnotesize {
   \item \noindent Integrate phy-less port \href{https://www.marvell.com/switching/link-street/}{\textbf{Marvell 88E6352 Switch}} with \href{http://www.nxp.com/products/reference-designs/qoriq-ls1021a-iot-gateway-reference-design:LS1021A-IoT}{\textbf{\underline{LS1021A-IOT}}} over PHY-less port.
   \item \noindent High Priority scheduling for selected TCP/UDP packets with Round Robin mechanism.
   \item \noindent Understand the complete network packet flow, layer 2 and working of Network Switch.
   \item \noindent Experience in Ethernet driver, socket buffers i.e. skb packets to prioritize UDP packets.
   \item \noindent Tailored device tree and gianfar linux driver.
}
\end{itemize}\vspace{-\topsep}%
\sectionsep       %%% Followed per Deedy resume

%
%
% Company
%
%
\MainSubHeading{\CompanyIII{}}{| Technical Leader}
\MainSubHeadingNote{\CompanyIIIexp{}}
\RoleDescription{%
   Gained experience on Ethernet, Wireless domain, linux kernel and system applications.
}
\vspace{-\topsep}\vspace{2pt}    % Remove any vertical spaced added by itemsep
\begin{itemize}\itemsep\isep
\footnotesize {
   \item \noindent OTA upgrade - Implemented failsafe mechanism.
   \item \noindent Integrated Atheros PHY with Marvell SOC \textendash{} MDIO protocol.
   \item \noindent Wireless Client aborts communication in 5GHz B.W. upon RADAR detection on \\ current DFS channel and raise CAC to station - first ever design for Client.
}
\end{itemize}\vspace{-\topsep}%
\sectionsep       %%% Followed per Deedy resume


%
%
% Company
%
%
\MainSubHeading{\CompanyII{}}{| Senior Engineer}
\MainSubHeadingNote{\CompanyIIexp{}}
\RoleDescription{%
   Worked in boot domain and experienced with Jenkins.
}
\vspace{-\topsep}\vspace{2pt}    % Remove any vertical spaced added by itemsep
\begin{itemize}\itemsep\isep
\footnotesize {
   \item \noindent Automate flashing \& loading of Wi-Fi FW binary from HOST over TFTP in boot terminal.
   \item \noindent Jenkins to automate Ethernet test-cases
}
\end{itemize}\vspace{-\topsep}%
\sectionsep


%
%
% Company
%
%
\MainSubHeading{\CompanyI{}}{| Senior Engineer}
\MainSubHeadingNote{\CompanyIexp{}}
\RoleDescription{%
   Responsible for Board Bring up, linux device driver, booting and system level programming. Experienced on I2C, SPI, MDIO, Ethernet, Wireless, Bluetooth, DDR and integrating memories i.e. NAND, eMMC with HOST SOC's.
}
\vspace{-\topsep}\vspace{2pt}    % Remove any vertical spaced added by itemsep
\begin{itemize}\itemsep\isep
\footnotesize {
   \item \noindent Board Bring UP of multiple targets like TI's DM388, TDA2xx and Freescale i.MX6 series.
   \item \noindent Linux device driver to configure HOST DM388 in slave mode over I2C \& SPI bus.
   %\item \noindent System level programming.
   \item \noindent Porting of u-boot, linux kernel, device-tree \& System level programming.
   \item \noindent Developed Applications for hostapd \& wpa supplicant to configure Wi-Fi targets. 
}
\end{itemize}\vspace{-\topsep}%
\sectionsep


\end{document}
