%%%%%%%%%%%%%%%%%%%%%%%%%%%%%%%%%%%%%%%%%%%%%%%%%%
%
% Latex Template - Tex file
% Compile with xelatex
% Version 1.0
% Shared under GPLv3
%
%%%%%%%%%%%%%%%%%%%%
%
%
%%%%%%%%%%%%%%%%%%%%
%
%
%%%%%%%%%%%%%%%%%%%%%%%%%%%%%%%%%%%%%%%%%%%%%%%%%%
%

\documentclass[a4paper]{article} % use "letterpaper" if you're in the US

%%%%%
% Load Geometry package after \LoadClass{article} to make effect
%
%\usepackage[top=0.75in, bottom=0.75in, left=0.55in, right=0.85in]{geometry}
\usepackage[hmargin=1.25cm, vmargin=0.75cm]{geometry}
%%%%%

%%%
% Dump data for checking - \blindtext[1]
% \blindtext = paragraph of dummy text
% \Blindtext = multiple paragraphs (one page long) of dummy text
% \Blinddocument = entire document, If you specify documentclass
%      as report then \Blinddocument will generate the document
%      with multiple chapters, sections, subsections, paragraphs,
%       and itemized lists.
\usepackage{blindtext}
\usepackage{lipsum}              % Dummy data - lipsum[2-4]
%%%

%%% Debug
% Show layout paramters - \layout or \layout* after \begin{document}
\usepackage{layout}
% \usepackage{layouts}             % Illustrates minor details - Useful package

%%% Debug
% Show outerframe, Uncomment this   - Debug
\usepackage{showframe}

%%%%
\usepackage[draft]{flowfram}
%
% draft option will draw the bounding boxes for each defined frame.
%  Bottom right of each bounding box (except for the bounding box
%  denoting the typeblock), a marker will be shown in the form: [T:IDN;IDL]
%  -  draft option set all conditionals to true
%      \showtypeblocktrue  % Display the bounding box for the typeblock.
%      \showtypeblockfalse % Do not display the bounding box for the typeblock.
%      \showmarginstrue    % Display the bounding box for the margins.
%      \showmarginsfalse   % Do not display the bounding box for the margins.
%      \showframebboxtrue  % Display the bounding box for the frames.
%      \showframebboxfalse % Do not display the bounding box for the frames
%
% The text of the document environment will flow from one frame to the next in
%  order of definition.
%
% \newflowframe[page-list]{width}{height}{x}{y}[label]
%  - width, height - width, height of frame
%  - x, y - Position of the bottom left hand corner of the frame relative to the
%    bottom left hand corner of the typeblock.
%
%  - page-list - optional argument, indicates the list of pages for which this
%    frame is defined. A page list can either be specified by the
%    keywords: all, odd, even or none, or by a comma-separated list of either
%    individual page numbers or page ranges. If page-list is omitted, all assumed.
%
\def\LEFTCOLUMNWIDTH  {0.65\textwidth}            % 2/3 of text width
\def\RIGHTCOLUMNWIDTH {0.32\textwidth}            % 1/3 of text width
\def\SIDEBYSIDEGAP {0.68\textwidth}

\newcommand{\LeftColWdth}{0.66\textwidth}
\newcommand{\RightColWdth}{0.33\textwidth}

%  Page 1 - Left and Right layout
%   Frame 1
\newflowframe[1]{\LEFTCOLUMNWIDTH}{\textheight}{0pt}{0pt}[left]
%   Frame 2
\newflowframe[1]{\RIGHTCOLUMNWIDTH}{\textheight}{\SIDEBYSIDEGAP}{0pt}[right]

%%%
% Modify flowframe attribute
% \setflowframe{ID-list}{key=value list}
%   ID-list can be found with draft option
%   Frame ID allocated in order it is defined
\setflowframe{1}{label=hi, backcolor=red}
\setflowframe{2}{backcolor=yellow}

% --- Begin Document ---
\begin{document}
%\flowframeshowlayout
%\layout                    % Debug

\lipsum[1-3]
\framebreak
\lipsum[4-9]

\end{document}
